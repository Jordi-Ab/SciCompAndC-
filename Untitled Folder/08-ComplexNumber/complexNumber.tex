\documentclass[8pt, compress]{beamer}

\usetheme{m}

\usepackage{booktabs}
\usepackage{multirow}
\usepackage{amsmath}

\usepackage{listings}
\usepackage{caption}
\definecolor{LightGrey}{rgb}{0.9629411,0.9629411,0.9629411}
\definecolor{LighterGrey}{gray}{0.99}
\definecolor{Mauve}{rgb}{0.58,0,0.82}
\definecolor{Emerald}{rgb}{0.31, 0.78, 0.47}
\definecolor{RoyalBlue}{rgb}{0.25, 0.41, 0.88}
\definecolor{myGreen}{cmyk}{0.82,0.11,1,0.25}

\lstset{
    language=C++,
    keywordstyle=\color{RoyalBlue},
    basicstyle=\scriptsize\ttfamily,
    %commentstyle=\color{Emerald}\scriptsize\ttfamily,
    morekeywords={define,ifndef,endif},
    commentstyle=\color{myGreen}\scriptsize\ttfamily,
    directivestyle=\color{Mauve}\scriptsize\ttfamily,
    showspaces=false,            
    showstringspaces=false,
    stringstyle=\color{Mauve}\scriptsize\ttfamily,
    numbers=left,
    numberstyle=\scriptsize,
    stepnumber=1,
    numbersep=8pt,
    showstringspaces=false,
    breaklines=true,
    frameround=ftff,
    frame=lines,
    backgroundcolor=\color{LightGrey}
} 
\def\inline{\lstinline[basicstyle=\ttfamily,keywordstyle={},directivestyle={}]}
\setbeamerfont{caption}{size=\tiny}

\newcommand*{\lsref}[1]{Listing~\ref{#1}}

% \usemintedstyle{trac}

\graphicspath{{./Figures/}}

\title{Building a class of complex numbers}
\subtitle{}
\date{}
\author{Daniele Avitabile}
\institute{University of Nottingham}

\begin{document}

\maketitle

\begin{frame}[fragile]
  \frametitle{Learning objectives}
  In this unit we will build our first class in small incremental steps. The
  directory \inline|Codes| contains material that will be used in this exercise,
  as well as solutions to questions. All material can be downloaded from our Moodle
  page.
  
  \begin{alertblock}{In this unit we will}

  \begin{enumerate}
    \item Recall fundamental concepts of a class
    \item Familiarise with the design of a class
    \item Translate mathematical operations into class members
    \item Return objects as method outputs
    \item Learn to override operators
  \end{enumerate}

  \end{alertblock}

\end{frame}

\section{Question 1}

\begin{frame}[fragile]
  \frametitle{Question 1}
  \begin{alertblock}{Question 1}
    The class \inline|ComplexNumber| written in the next slide contains the
    following members
  \begin{enumerate}
    \item Two double precision floating point variable \inline |mRealPart|,
      \inline|mImaginaryPart|.
    \item An overridden default constructor \inline|ComplexNumber()| which sets
      real and imaginary parts to $0$.
    \item An overridden insertion operator \inline|<<| to print formatted
      output (recall Section 6.3 in Pitt-Francis\&Whiteley).
  \end{enumerate}

    Familiarise yourself with \inline|ComplexNumber|, download the corresponding
    code and write a \inline|Driver.cpp| file that produces the following output
    \begin{lstlisting}[numbers=none]
Printing the complex number z1 = (0 + 0i)
    \end{lstlisting}
  \end{alertblock}
\end{frame}

\begin{frame}[fragile]
  \frametitle{The initial class ComplexNumber}
  \begin{columns}[t]
    \column{0.5\textwidth}
    \lstinputlisting[caption=Codes/ComplexNumber1/ComplexNumber.hpp,
                     numbers=none]{Codes/ComplexNumber1/ComplexNumber.hpp} 

    \column{0.5\textwidth}
    \lstinputlisting[caption=Codes/ComplexNumber1/ComplexNumber.cpp,
                     numbers=none]{Codes/ComplexNumber1/ComplexNumber.cpp} 

  \end{columns}
\end{frame}

\section{Question 2}
\begin{frame}[fragile]
  \frametitle{Question 2}
  \begin{alertblock}{Question 2 (attempt it before reading the next slide)}
  Let $z= r e^{i \theta}$ be a complex number. Add and test the following members to
  \inline|ComplexNumber|:
  \begin{enumerate}
    \item A constructor that initialises $\text{Re}(z)$ and $\text{Im}(x)$ (use
      this prototype)
      \lstinputlisting[numbers=none, linerange=12-12]{ Codes/ComplexNumber2/ComplexNumber.hpp} 

    \item A method that returns the modulus $r$ of $z$ (use this prototype)
      \lstinputlisting[numbers=none, linerange=15-15]{ Codes/ComplexNumber2/ComplexNumber.hpp} 

    \item A method that returns the argument $\theta$ of $z$ (use this prototype)
      \lstinputlisting[numbers=none, linerange=18-18]{ Codes/ComplexNumber2/ComplexNumber.hpp} 

    \item A method that returns the complex number $z^n$. You should use the de
      Moivre's identity $z^n = (r e^{i\theta})^n 
                             = r^n [ \cos(n\theta) + i \sin(n \theta) ]$, the
      two methods \inline|ComputeModulus|, \inline|ComputeArgument| and the
      following prototype
      \lstinputlisting[numbers=none, linerange=21-21]{ Codes/ComplexNumber2/ComplexNumber.hpp} 
  \end{enumerate}

  \end{alertblock}
\end{frame}

\begin{frame}[fragile]
  \frametitle{Comments on CalculatePower}

  \lstinputlisting[caption=From Codes/ComplexNumber2/ComplexNumber.cpp,
                   linerange=30-45]{Codes/ComplexNumber2/ComplexNumber.cpp} 

  \begin{alertblock}{Things to note}
    \begin{description}
      \item[lines 5,6] \inline|CalculateModulus| and \inline|CalculateArgument|
	are accessible within \inline|ComplexNumber|, hence we use them here.

      \item[line 13] We return an instance of \inline|ComplexNumber| (the
	complex number that contains $z^n$)
    \end{description}
  \end{alertblock}
\end{frame}

\section{Overloading operators}
\begin{frame}[fragile]
  \frametitle{Overloading operators}
  Let $u,v,z \in \mathbb{C}$. The following expressions are mathematically well-defined
  \[
    z = u, \quad z = -u, \quad z = u + w, \quad z = u - v,
  \]
  so it would seem natural to perform similar operations in our codes
  \begin{lstlisting}[numbers=none]
ComplexNumber u(1.0,2.0);
ComplexNumber w(3.0,4.0);
ComplexNumber z;
z = u; 
z = -u; 
z = u + w;
z = u - w;
  \end{lstlisting}

  Unfortunately, the operators \inline|=|, \inline|+|, and \inline|-| are well
  defined for standard variables (for instance \inline|double| or \inline|int|) but
  they are not defined for a \inline|ComplexNumber|, hence the code above would
  produce a compilation error.

  However, \inline|C++| allows to overcome this difficulty using \alert{operator
  overloading}. For instance we can attribute a meaning to the expression
  \inline|z=u| whenever \inline|u| and \inline|z| are \inline|ComplexNumber|.
\end{frame}

\begin{frame}[fragile]
  \frametitle{Overloading the assignment operator \inline|=|}

  We begin by overloading the assignment operator \inline|=|, which will be used to
  copy the content of \inline|u| into \inline|z|
  \lstinputlisting[numbers=none, linerange={8-8,11-12}]{Codes/ComplexNumber3/Driver.cpp} 

  hence we add the following method to \inline|ComplexNumber.hpp|
  \lstinputlisting[numbers=none,
                   linerange=24-24]{Codes/ComplexNumber3/ComplexNumber.hpp} 

  and implement it in the source file \inline|ComplexNumber.cpp|
  \lstinputlisting[numbers=none,
                   linerange=47-53]{Codes/ComplexNumber3/ComplexNumber.cpp} 

  This method deserves a careful analysis, so we discuss it in depth in the next
  slide
\end{frame}

\begin{frame}[fragile]
  \frametitle{Overloading the assignment operator \inline|=|}
  \vspace{-1em}
  \lstinputlisting[linerange=47-53,
                   caption=From Codes/ComplexNumber3/ComplexNumber.cpp]{Codes/ComplexNumber3/ComplexNumber.cpp} 
  \begin{alertblock}{Things to note}
    \begin{description}
      \item[Line 1] This method returns a reference to an instance of the class.
      \item[Line 2] The argument of this method is a \alert{reference} to
	another instance of the class. This is because, by default, all method
	arguments are called by copy. The use of \inline|const| guarantees that
	the argument won't be modified.
      \item[Lines 4-5] The real and imaginary part of the argument are
	``copied'' to the private member of the class.
      \item[Line 6] Every \inline|C++| object has access to its own address
	through an important pointer called \inline|this|. Here we return the
	content of \inline|this|, the current complex number.
    \end{description}
  \end{alertblock}

\end{frame}

\begin{frame}[fragile]
  \frametitle{Using the assignment operator \inline|=|}

  Let us use the newly defined assignment operator
  \lstinputlisting[numbers=none, linerange={8-8, 11-14}]{Codes/ComplexNumber3/Driver.cpp} 
  When the code above is executed, 
  \begin{enumerate}
    \item An object \inline|u| is instantiated using one of the constructors. The
      object contains the number $1.4 + 2.2 i$.
    \item An object \inline|z| is instantiated with the overloaded default
      constructor, so it contains the number $0 + 0i$.
    \item The object \inline|u| is passed as an argument to the method
      \inline|=| of the object \inline|z|, that is, the pointer \inline|this| in
      the previous slide contains the address of \inline|z|.
    \item The content of \inline|u| is copied into \inline|z|.
  \end{enumerate}

  We obtain the following output
  \begin{lstlisting}[numbers=none]
u = (1.4 +2.2i)
z = (1.4 +2.2i)
  \end{lstlisting}

  See the full implementation in \inline|Codes/ComplexNumber/ComplexNumber3|
\end{frame}

\section{Question 3}

\begin{frame}[fragile]
  \frametitle{Question 3}
  \vspace{-2em}
  \begin{alertblock}{Question 3}
  Let $u,v,z \in \mathbb{C}$. Download the code in
  \inline|Codes/ComplexNumber/ComplexNumber3| and add the following members to
  \inline|ComplexNumber|
  % \[
  % z = -u, \qquad z = u + v, \qquad z = u - v
  % \]
  
  \begin{enumerate}
    \item A method that overloads the unary subtraction operator \inline|-|, in
      order to perform the mathematical operation $z=-u$ (use this prototype)
      \lstinputlisting[numbers=none, linerange=27-27]{Codes/ComplexNumber4/ComplexNumber.hpp} 

    \item A method that overloads the binary addition operator \inline|+|, in
      order to perform the mathematical operation $z=u+v$ (use this prototype)
      \lstinputlisting[numbers=none, linerange=30-30]{ Codes/ComplexNumber4/ComplexNumber.hpp} 

    \item A method that overloads the binary addition operator \inline|-|, in
      order to perform the mathematical operation $z=u-v$ (use this prototype)
      \lstinputlisting[numbers=none, linerange=33-33]{ Codes/ComplexNumber4/ComplexNumber.hpp} 
  \end{enumerate}

  Write a file \inline|Driver.cpp| to test your class. You now have a fully
  functional \inline|ComplexNumber| class whose modularity we will exploit in future
  units!

  \end{alertblock}
\end{frame}

\end{document}
